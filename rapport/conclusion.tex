
\chapter{Conclusion}

Nous avons vu dans ce projet différentes techniques que l'on peut utiliser 
pour la reconnaissance de caractères:
\begin{itemize}
  \item les réseaux de neurones \nfw{fully-connected};
  \item les algorithmes classiques de \nfw{machine learning} : \nfw{random forest}, \nfw{k-nearest neighbors}, \nfw{gradient boosting}, et autres;
  \item les réseaux convolutifs.
\end{itemize}

Nous avons abordé le problème à la fois en effectuant du \nfw{feature engineering} sur 
les données mais également en donnant les images brutes en entrées des réseaux. 

Si les réseaux convolutifs donnent clairement les meilleurs performances 
(et peuvent être entraînés rapidement), on obtient tout de même des taux 
de reconnaissance satisfaisants avec les techniques utilisant des features
relativement simples.

L'un des problèmes rencontrés est celui de la difficulté de généralisation 
après l'apprentissage.
En l'absence d'augmentation du jeu d'entraînement, le taux de reconnaissance 
sur des chiffres écrits par l'utilisateur est décevant.
Ceci montre qu'en plus d'utiliser de bons algorithmes, il faut surtout avoir 
un jeu de données le plus complet possible pour espérer pouvoir faire 
une classification satisfaisante par la suite.



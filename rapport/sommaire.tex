
\chapter{Sommaire}

Le but de ce projet est d'étudier l'utilisation de réseaux de neurones 
profonds pour la reconnaissance de caractères manuscrits. 
Différentes méthodes de classifications seront utilisées et comparées à un réseau 
de neurones simple et à des algorithmes n'utilisant pas de réseaux de neurones.
Le langage \Python sera utilisé pour l'écriture des programmes, ce dernier 
fournissant bon nombre de bibliothèques de \nfw{machine learning}. \\
Pour mieux comprendre les algorithmes et idées mis en jeux, une brève 
introduction aux principes des réseaux de neurones sera présentée. 
%Une interface graphique permettant de mieux visualiser les résultats sera 
%également présentée.

\vspace{1cm}

Mots-clés: réseaux de neurones, \nfw{machine learning}, \nfw{deep networks}, reconnaissance de caractères, \Python.

\vfill


\chapter{Transformée de Fourier de l'image}

On se propose dans ce qui suit d'expliquer les calculs effectués 
par la fonction \tcode{rfft} présentées dans le chapitre \ref{chap:features}.

\renewcommand{\Re}{\operatorname{Re}}
\renewcommand{\Im}{\operatorname{Im}}
\let\conjugatet\overline

Soit $x = (x_0, \cdots, x_{2n-1})$ un vecteur réel de taille $2n$.

On définie la transformée de Fourier discrète de $x$ par un vecteur 
$y$ de même taille dont les composantes sont:
\[
y_j = \sum_{k = 0}^{2n-1} x_k \exp\left(-\frac{2ijk\pi}{2n}\right)
\]

On remarque que $y_j = \operatorname{Conj} y_{2n-j}$. 
Ainsi, seules les valeurs $y_j$ pour $j \leq n$ sont utiles.
De plus, pour $j = n$, on obtient $y_n = \operatorname{Conj} y_{n}$, 
donc la partie imaginaire de $y_n$ est nulle.
Enfin, $y_0$ est un réel.
Ainsi, on peut encoder toute l'information contenue 
dans les $y_j$ pour $j$ de $0$ à $2n-1$ dans de vecteur réel 
$f(x)$ de taille $2n$ définit ci-après.
\[
f(x) = 
\begin{pmatrix}
  y_0 \\
  \Re(y_1) \\
  \Im(y_1) \\
  \vdots \\
  \Re(y_{n-1}) \\
  \Im(y_{n-1}) \\
  \Re(y_n)
\end{pmatrix}
\]

Il est possible de reconstruire $x$ à partir de sa transformée.

\begin{align*}
x^{*}_j &= \frac{1}{2n} \sum_{k=0}^{2n-1} y_k \exp\left( \frac{2ijk\pi}{2n} \right)  \\
        &= \frac{1}{2n} \sum_{k=0}^{2n-1} \sum_{l = 0}^{2n-1} x_l \exp\left(-\frac{2ikl\pi}{2n}\right) \exp\left( \frac{2ijk\pi}{2n} \right) \\
		&= \frac{1}{2n} \sum_{l=0}^{2n-1} \sum_{k = 0}^{2n-1} x_l \exp\left(\frac{2ik(j-l)\pi}{2n}\right) 
\end{align*}

Pour $l = j$, les termes de la seconde somme valent tous $x_l$ et il y en a 
$2n$, on se retrouve donc avec $2n x_l$. \\
Pour $l \neq j$, on a la somme suivante:
\[
x_l \sum_{k = 0}^{2n-1} \exp\left(\frac{2i(j-l)\pi}{2n}\right)^k 
\]
qui est une somme géométrique valant
\[
x_l \times \frac{1-\exp\left(\frac{2i(j-l)\pi}{2n}\right)^{2n}}{1-\exp\left(\frac{2i(j-l)\pi}{2n}\right)} = 0
\]
(que l'on peut également voir comme une somme des racines $2n$-ème de l'unité)

Au final, $x^{*}_j = x_j$.

Une autre écriture de $x^{*}_j$, que l'on utilisera en pratique, est donnée par:

\[
x^{*}_j = \frac{1}{2n} \left[ \sum_{k=1}^{n-1} \left[ y_k \exp\left( \frac{2ijk\pi}{2n} \right) 
          + \overline{y_k} \exp\left(-\frac{2ijk\pi}{2n} \right) \right]
		  + y_0 + (-1)^{j} y_n \right]
\]


Considérons maintenant une matrice $X$ carrée d'ordre $2n$. 
De la même manière que l'on avait vectorisé les fonctions 
d'activation dans la partie concernant les réseaux de neurones, 
on peut vectoriser $f$ et l'appliquer à la matrice $X$. 
On définit donc $f(X)$ comme la matrice dont les colonnes sont 
les images des colonnes de $X$ par $f$.

Le passage d'une image $X$ de dimensions paires dans le domaine 
de Fourier se fait en appliquant la transformée une fois sur les 
colonnes puis une fois sur les lignes. 
La fonction \tcode{rfft} de \tcode{scipy} calcule $f$.
\[
Y = f(f(X)^T)^T
\]

Une implémentation de cette fonction est donnée en \Python dans le 
fichier \tcode{features.py} sous le nom \tcode{rfft_vect}; cette 
fonction n'est pas utilisée en pratique pour des raisons de performance.

\chapter{Introduction}

La reconnaissance de caractères, ou plus généralement de textes manuscrits, est 
un domaine en pleine expansion. 
De nombreux géants de l'informatique comme Google, Apple ou Microsoft se battent pour 
offrir à leurs utilisateurs (et probablement à d'autres fins) le meilleur système 
de reconnaissance de texte. Les utilisations sont nombreuses :
\begin{itemize}
  \item application de prises de notes;
  \item saisie de formulaires;
  \item saisie d'une adresse pour le système GPS de certaines voitures;
  \item reconnaissance d'adresse postale sur les enveloppes;
  \item reconnaissance d'un montant sur les chéques.
\end{itemize}

\vspace{1em}

Dans le cadre de ce projet, on se contentera de tenter de reconnaître des chiffres. 
On utilisera pour cela le jeu de données MNIST, qui est un jeu de données contenant 
60000 images d'apprentissage et 10000 images de test. Ces images, représentants des 
chiffres écrits à la main, sont en niveaux de gris et font 28 pixels par 28 pixels.
Elles sont issues de deux jeux de données collectés par le NIST 
(United States' National Institute of Standards and Technology). 
A chaque image est associé un label (i.e.\/ le chiffre représenté). 
Ce jeu de données est pratiquement devenu un standard dans le monde de la 
reconnaissance de caractères.

Ce rapport est découpé en plusieurs parties. Dans un premier temps, une introduction 
sur les réseaux de neurones permettra de se familiariser avec ces derniers. 
On sera alors en mesure d'écrire nous-même les algorithmes pour les réseaux les 
plus simples et nous pourrons construire un premier réseau permettant de répondre 
au problème de classification que l'on se pose. \\
Ensuite, nous pourrons mettre en oeuvre des techniques de`\nfw{machine learning} 
plus poussés et plus adaptés à notre domaine. On commencera pour cela par 
étudier certaines techniques propres au problème de reconnaissance de caractères
(l'extraction de \nfw{features}).
Enfin, nous nous pencherons sur l'utilisation de réseaux de neurones convolutionnels, 
ces derniers étant particulièrement adaptés aux problèmes de vision par ordinateur 
et de traitement d'image.
